\documentclass[12pt]{standalone}
\usepackage{tikz}
\usepackage{tikz-3dplot}


\usepackage{anyfontsize}
\renewcommand{\normalsize}{\fontsize{14pt}{16pt}\selectfont}


\begin{document}

\tdplotsetmaincoords{60}{110}
\begin{tikzpicture}[tdplot_main_coords, scale=0.8]

% Define coordinate system
\coordinate (O) at (0,0,0);

% Draw axes
\draw[thick,->] (0,0,0) -- (12,0,0) node[anchor=north east]{$x_1$};
\draw[thick,->] (0,0,0) -- (0,12,0) node[anchor=north west]{$x_2$};
\draw[thick,->] (0,0,0) -- (0,0,12) node[anchor=south]{$x_3$};

% Key vertices of the feasible region
\coordinate (A) at (0,0,0);      % Origin
\coordinate (B) at (10,0,0);     % x1-axis intercept of constraint 2
\coordinate (C) at (0,10,0);     % x2-axis intercept of constraints 1&3
\coordinate (D) at (0,0,10);     % x3-axis intercept of constraints 1&2
\coordinate (E) at (4,4,4);      % Intersection of all three constraints

% Draw the polyhedron faces
% Face 1: Triangle on x1=0 plane (origin, (0,10,0), (0,0,10))
\fill[blue!30,opacity=0.8] (A) -- (C) -- (D) -- cycle;

% Face 2: Triangle on x2=0 plane (origin, (10,0,0), (0,0,10))
\fill[red!30,opacity=0.8] (A) -- (B) -- (D) -- cycle;

% Face 3: Triangle on x3=0 plane (origin, (10,0,0), (0,10,0))
\fill[green!30,opacity=0.8] (A) -- (B) -- (C) -- cycle;

% Face 4: Triangle from constraint intersection ((10,0,0), (0,10,0), (4,4,4))
\fill[yellow!30,opacity=0.8] (B) -- (C) -- (E) -- cycle;

% Face 5: Triangle from constraint intersection ((10,0,0), (0,0,10), (4,4,4))
\fill[orange!30,opacity=0.8] (B) -- (D) -- (E) -- cycle;

% Face 6: Triangle from constraint intersection ((0,10,0), (0,0,10), (4,4,4))
\fill[purple!30,opacity=0.8] (C) -- (D) -- (E) -- cycle;

% Draw edges of the polyhedron
\draw[thick,dashed] (A) -- (B);
\draw[thick,dashed] (A) -- (C);
\draw[thick,dashed] (A) -- (D);
\draw[thick] (B) -- (C);
\draw[thick] (B) -- (D);
\draw[thick] (C) -- (D);
\draw[thick] (B) -- (E);
\draw[thick] (C) -- (E);
\draw[thick] (D) -- (E);

% Mark vertices
\fill[black] (A) circle (2pt) node[left] {$(0,0,0)$};
\fill[black] (B) circle (2pt) node[left] {$(10,0,0)$};
\fill[black] (C) circle (2pt) node[below] {$(0,10,0)$};
\fill[black] (D) circle (2pt) node[left] {$(0,0,10)$};
\fill[purple] (E) circle (3pt) node[left] {$(4,4,4)$};

\end{tikzpicture}

\end{document}
